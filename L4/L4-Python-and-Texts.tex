
% Default to the notebook output style

    


% Inherit from the specified cell style.




    
\documentclass{article}

    
    \usepackage[adobefonts]{ctex}
    \usepackage{graphicx} % Used to insert images
    \usepackage{adjustbox} % Used to constrain images to a maximum size 
    \usepackage{color} % Allow colors to be defined
    \usepackage{enumerate} % Needed for markdown enumerations to work
    \usepackage{geometry} % Used to adjust the document margins
    \usepackage{amsmath} % Equations
    \usepackage{amssymb} % Equations
    \usepackage{eurosym} % defines \euro
    \usepackage[mathletters]{ucs} % Extended unicode (utf-8) support
    \usepackage[utf8x]{inputenc} % Allow utf-8 characters in the tex document
    \usepackage{fancyvrb} % verbatim replacement that allows latex
    \usepackage{grffile} % extends the file name processing of package graphics 
                         % to support a larger range 
    % The hyperref package gives us a pdf with properly built
    % internal navigation ('pdf bookmarks' for the table of contents,
    % internal cross-reference links, web links for URLs, etc.)
    \usepackage{hyperref}
    \usepackage{longtable} % longtable support required by pandoc >1.10
    \usepackage{booktabs}  % table support for pandoc > 1.12.2
    

    
    
    \definecolor{orange}{cmyk}{0,0.4,0.8,0.2}
    \definecolor{darkorange}{rgb}{.71,0.21,0.01}
    \definecolor{darkgreen}{rgb}{.12,.54,.11}
    \definecolor{myteal}{rgb}{.26, .44, .56}
    \definecolor{gray}{gray}{0.45}
    \definecolor{lightgray}{gray}{.95}
    \definecolor{mediumgray}{gray}{.8}
    \definecolor{inputbackground}{rgb}{.95, .95, .85}
    \definecolor{outputbackground}{rgb}{.95, .95, .95}
    \definecolor{traceback}{rgb}{1, .95, .95}
    % ansi colors
    \definecolor{red}{rgb}{.6,0,0}
    \definecolor{green}{rgb}{0,.65,0}
    \definecolor{brown}{rgb}{0.6,0.6,0}
    \definecolor{blue}{rgb}{0,.145,.698}
    \definecolor{purple}{rgb}{.698,.145,.698}
    \definecolor{cyan}{rgb}{0,.698,.698}
    \definecolor{lightgray}{gray}{0.5}
    
    % bright ansi colors
    \definecolor{darkgray}{gray}{0.25}
    \definecolor{lightred}{rgb}{1.0,0.39,0.28}
    \definecolor{lightgreen}{rgb}{0.48,0.99,0.0}
    \definecolor{lightblue}{rgb}{0.53,0.81,0.92}
    \definecolor{lightpurple}{rgb}{0.87,0.63,0.87}
    \definecolor{lightcyan}{rgb}{0.5,1.0,0.83}
    
    % commands and environments needed by pandoc snippets
    % extracted from the output of `pandoc -s`
    \providecommand{\tightlist}{%
      \setlength{\itemsep}{0pt}\setlength{\parskip}{0pt}}
    \DefineVerbatimEnvironment{Highlighting}{Verbatim}{commandchars=\\\{\}}
    % Add ',fontsize=\small' for more characters per line
    \newenvironment{Shaded}{}{}
    \newcommand{\KeywordTok}[1]{\textcolor[rgb]{0.00,0.44,0.13}{\textbf{{#1}}}}
    \newcommand{\DataTypeTok}[1]{\textcolor[rgb]{0.56,0.13,0.00}{{#1}}}
    \newcommand{\DecValTok}[1]{\textcolor[rgb]{0.25,0.63,0.44}{{#1}}}
    \newcommand{\BaseNTok}[1]{\textcolor[rgb]{0.25,0.63,0.44}{{#1}}}
    \newcommand{\FloatTok}[1]{\textcolor[rgb]{0.25,0.63,0.44}{{#1}}}
    \newcommand{\CharTok}[1]{\textcolor[rgb]{0.25,0.44,0.63}{{#1}}}
    \newcommand{\StringTok}[1]{\textcolor[rgb]{0.25,0.44,0.63}{{#1}}}
    \newcommand{\CommentTok}[1]{\textcolor[rgb]{0.38,0.63,0.69}{\textit{{#1}}}}
    \newcommand{\OtherTok}[1]{\textcolor[rgb]{0.00,0.44,0.13}{{#1}}}
    \newcommand{\AlertTok}[1]{\textcolor[rgb]{1.00,0.00,0.00}{\textbf{{#1}}}}
    \newcommand{\FunctionTok}[1]{\textcolor[rgb]{0.02,0.16,0.49}{{#1}}}
    \newcommand{\RegionMarkerTok}[1]{{#1}}
    \newcommand{\ErrorTok}[1]{\textcolor[rgb]{1.00,0.00,0.00}{\textbf{{#1}}}}
    \newcommand{\NormalTok}[1]{{#1}}
    
    % Define a nice break command that doesn't care if a line doesn't already
    % exist.
    \def\br{\hspace*{\fill} \\* }
    % Math Jax compatability definitions
    \def\gt{>}
    \def\lt{<}
    % Document parameters
    \title{L4-Python-and-Texts}
    
    
    

    % Pygments definitions
    
\makeatletter
\def\PY@reset{\let\PY@it=\relax \let\PY@bf=\relax%
    \let\PY@ul=\relax \let\PY@tc=\relax%
    \let\PY@bc=\relax \let\PY@ff=\relax}
\def\PY@tok#1{\csname PY@tok@#1\endcsname}
\def\PY@toks#1+{\ifx\relax#1\empty\else%
    \PY@tok{#1}\expandafter\PY@toks\fi}
\def\PY@do#1{\PY@bc{\PY@tc{\PY@ul{%
    \PY@it{\PY@bf{\PY@ff{#1}}}}}}}
\def\PY#1#2{\PY@reset\PY@toks#1+\relax+\PY@do{#2}}

\expandafter\def\csname PY@tok@sx\endcsname{\def\PY@tc##1{\textcolor[rgb]{0.00,0.50,0.00}{##1}}}
\expandafter\def\csname PY@tok@ni\endcsname{\let\PY@bf=\textbf\def\PY@tc##1{\textcolor[rgb]{0.60,0.60,0.60}{##1}}}
\expandafter\def\csname PY@tok@ge\endcsname{\let\PY@it=\textit}
\expandafter\def\csname PY@tok@cm\endcsname{\let\PY@it=\textit\def\PY@tc##1{\textcolor[rgb]{0.25,0.50,0.50}{##1}}}
\expandafter\def\csname PY@tok@nn\endcsname{\let\PY@bf=\textbf\def\PY@tc##1{\textcolor[rgb]{0.00,0.00,1.00}{##1}}}
\expandafter\def\csname PY@tok@gp\endcsname{\let\PY@bf=\textbf\def\PY@tc##1{\textcolor[rgb]{0.00,0.00,0.50}{##1}}}
\expandafter\def\csname PY@tok@sb\endcsname{\def\PY@tc##1{\textcolor[rgb]{0.73,0.13,0.13}{##1}}}
\expandafter\def\csname PY@tok@vc\endcsname{\def\PY@tc##1{\textcolor[rgb]{0.10,0.09,0.49}{##1}}}
\expandafter\def\csname PY@tok@cs\endcsname{\let\PY@it=\textit\def\PY@tc##1{\textcolor[rgb]{0.25,0.50,0.50}{##1}}}
\expandafter\def\csname PY@tok@gs\endcsname{\let\PY@bf=\textbf}
\expandafter\def\csname PY@tok@k\endcsname{\let\PY@bf=\textbf\def\PY@tc##1{\textcolor[rgb]{0.00,0.50,0.00}{##1}}}
\expandafter\def\csname PY@tok@kn\endcsname{\let\PY@bf=\textbf\def\PY@tc##1{\textcolor[rgb]{0.00,0.50,0.00}{##1}}}
\expandafter\def\csname PY@tok@c\endcsname{\let\PY@it=\textit\def\PY@tc##1{\textcolor[rgb]{0.25,0.50,0.50}{##1}}}
\expandafter\def\csname PY@tok@o\endcsname{\def\PY@tc##1{\textcolor[rgb]{0.40,0.40,0.40}{##1}}}
\expandafter\def\csname PY@tok@err\endcsname{\def\PY@bc##1{\setlength{\fboxsep}{0pt}\fcolorbox[rgb]{1.00,0.00,0.00}{1,1,1}{\strut ##1}}}
\expandafter\def\csname PY@tok@go\endcsname{\def\PY@tc##1{\textcolor[rgb]{0.53,0.53,0.53}{##1}}}
\expandafter\def\csname PY@tok@il\endcsname{\def\PY@tc##1{\textcolor[rgb]{0.40,0.40,0.40}{##1}}}
\expandafter\def\csname PY@tok@kr\endcsname{\let\PY@bf=\textbf\def\PY@tc##1{\textcolor[rgb]{0.00,0.50,0.00}{##1}}}
\expandafter\def\csname PY@tok@s2\endcsname{\def\PY@tc##1{\textcolor[rgb]{0.73,0.13,0.13}{##1}}}
\expandafter\def\csname PY@tok@sr\endcsname{\def\PY@tc##1{\textcolor[rgb]{0.73,0.40,0.53}{##1}}}
\expandafter\def\csname PY@tok@ss\endcsname{\def\PY@tc##1{\textcolor[rgb]{0.10,0.09,0.49}{##1}}}
\expandafter\def\csname PY@tok@s1\endcsname{\def\PY@tc##1{\textcolor[rgb]{0.73,0.13,0.13}{##1}}}
\expandafter\def\csname PY@tok@nl\endcsname{\def\PY@tc##1{\textcolor[rgb]{0.63,0.63,0.00}{##1}}}
\expandafter\def\csname PY@tok@s\endcsname{\def\PY@tc##1{\textcolor[rgb]{0.73,0.13,0.13}{##1}}}
\expandafter\def\csname PY@tok@mf\endcsname{\def\PY@tc##1{\textcolor[rgb]{0.40,0.40,0.40}{##1}}}
\expandafter\def\csname PY@tok@mi\endcsname{\def\PY@tc##1{\textcolor[rgb]{0.40,0.40,0.40}{##1}}}
\expandafter\def\csname PY@tok@gi\endcsname{\def\PY@tc##1{\textcolor[rgb]{0.00,0.63,0.00}{##1}}}
\expandafter\def\csname PY@tok@nt\endcsname{\let\PY@bf=\textbf\def\PY@tc##1{\textcolor[rgb]{0.00,0.50,0.00}{##1}}}
\expandafter\def\csname PY@tok@nc\endcsname{\let\PY@bf=\textbf\def\PY@tc##1{\textcolor[rgb]{0.00,0.00,1.00}{##1}}}
\expandafter\def\csname PY@tok@no\endcsname{\def\PY@tc##1{\textcolor[rgb]{0.53,0.00,0.00}{##1}}}
\expandafter\def\csname PY@tok@vi\endcsname{\def\PY@tc##1{\textcolor[rgb]{0.10,0.09,0.49}{##1}}}
\expandafter\def\csname PY@tok@c1\endcsname{\let\PY@it=\textit\def\PY@tc##1{\textcolor[rgb]{0.25,0.50,0.50}{##1}}}
\expandafter\def\csname PY@tok@kt\endcsname{\def\PY@tc##1{\textcolor[rgb]{0.69,0.00,0.25}{##1}}}
\expandafter\def\csname PY@tok@bp\endcsname{\def\PY@tc##1{\textcolor[rgb]{0.00,0.50,0.00}{##1}}}
\expandafter\def\csname PY@tok@mh\endcsname{\def\PY@tc##1{\textcolor[rgb]{0.40,0.40,0.40}{##1}}}
\expandafter\def\csname PY@tok@se\endcsname{\let\PY@bf=\textbf\def\PY@tc##1{\textcolor[rgb]{0.73,0.40,0.13}{##1}}}
\expandafter\def\csname PY@tok@nv\endcsname{\def\PY@tc##1{\textcolor[rgb]{0.10,0.09,0.49}{##1}}}
\expandafter\def\csname PY@tok@sh\endcsname{\def\PY@tc##1{\textcolor[rgb]{0.73,0.13,0.13}{##1}}}
\expandafter\def\csname PY@tok@sc\endcsname{\def\PY@tc##1{\textcolor[rgb]{0.73,0.13,0.13}{##1}}}
\expandafter\def\csname PY@tok@gd\endcsname{\def\PY@tc##1{\textcolor[rgb]{0.63,0.00,0.00}{##1}}}
\expandafter\def\csname PY@tok@kp\endcsname{\def\PY@tc##1{\textcolor[rgb]{0.00,0.50,0.00}{##1}}}
\expandafter\def\csname PY@tok@mo\endcsname{\def\PY@tc##1{\textcolor[rgb]{0.40,0.40,0.40}{##1}}}
\expandafter\def\csname PY@tok@vg\endcsname{\def\PY@tc##1{\textcolor[rgb]{0.10,0.09,0.49}{##1}}}
\expandafter\def\csname PY@tok@si\endcsname{\let\PY@bf=\textbf\def\PY@tc##1{\textcolor[rgb]{0.73,0.40,0.53}{##1}}}
\expandafter\def\csname PY@tok@ow\endcsname{\let\PY@bf=\textbf\def\PY@tc##1{\textcolor[rgb]{0.67,0.13,1.00}{##1}}}
\expandafter\def\csname PY@tok@gu\endcsname{\let\PY@bf=\textbf\def\PY@tc##1{\textcolor[rgb]{0.50,0.00,0.50}{##1}}}
\expandafter\def\csname PY@tok@kd\endcsname{\let\PY@bf=\textbf\def\PY@tc##1{\textcolor[rgb]{0.00,0.50,0.00}{##1}}}
\expandafter\def\csname PY@tok@w\endcsname{\def\PY@tc##1{\textcolor[rgb]{0.73,0.73,0.73}{##1}}}
\expandafter\def\csname PY@tok@kc\endcsname{\let\PY@bf=\textbf\def\PY@tc##1{\textcolor[rgb]{0.00,0.50,0.00}{##1}}}
\expandafter\def\csname PY@tok@na\endcsname{\def\PY@tc##1{\textcolor[rgb]{0.49,0.56,0.16}{##1}}}
\expandafter\def\csname PY@tok@sd\endcsname{\let\PY@it=\textit\def\PY@tc##1{\textcolor[rgb]{0.73,0.13,0.13}{##1}}}
\expandafter\def\csname PY@tok@mb\endcsname{\def\PY@tc##1{\textcolor[rgb]{0.40,0.40,0.40}{##1}}}
\expandafter\def\csname PY@tok@gt\endcsname{\def\PY@tc##1{\textcolor[rgb]{0.00,0.27,0.87}{##1}}}
\expandafter\def\csname PY@tok@gr\endcsname{\def\PY@tc##1{\textcolor[rgb]{1.00,0.00,0.00}{##1}}}
\expandafter\def\csname PY@tok@m\endcsname{\def\PY@tc##1{\textcolor[rgb]{0.40,0.40,0.40}{##1}}}
\expandafter\def\csname PY@tok@nd\endcsname{\def\PY@tc##1{\textcolor[rgb]{0.67,0.13,1.00}{##1}}}
\expandafter\def\csname PY@tok@nb\endcsname{\def\PY@tc##1{\textcolor[rgb]{0.00,0.50,0.00}{##1}}}
\expandafter\def\csname PY@tok@nf\endcsname{\def\PY@tc##1{\textcolor[rgb]{0.00,0.00,1.00}{##1}}}
\expandafter\def\csname PY@tok@gh\endcsname{\let\PY@bf=\textbf\def\PY@tc##1{\textcolor[rgb]{0.00,0.00,0.50}{##1}}}
\expandafter\def\csname PY@tok@cp\endcsname{\def\PY@tc##1{\textcolor[rgb]{0.74,0.48,0.00}{##1}}}
\expandafter\def\csname PY@tok@ne\endcsname{\let\PY@bf=\textbf\def\PY@tc##1{\textcolor[rgb]{0.82,0.25,0.23}{##1}}}

\def\PYZbs{\char`\\}
\def\PYZus{\char`\_}
\def\PYZob{\char`\{}
\def\PYZcb{\char`\}}
\def\PYZca{\char`\^}
\def\PYZam{\char`\&}
\def\PYZlt{\char`\<}
\def\PYZgt{\char`\>}
\def\PYZsh{\char`\#}
\def\PYZpc{\char`\%}
\def\PYZdl{\char`\$}
\def\PYZhy{\char`\-}
\def\PYZsq{\char`\'}
\def\PYZdq{\char`\"}
\def\PYZti{\char`\~}
% for compatibility with earlier versions
\def\PYZat{@}
\def\PYZlb{[}
\def\PYZrb{]}
\makeatother


    % Exact colors from NB
    \definecolor{incolor}{rgb}{0.0, 0.0, 0.5}
    \definecolor{outcolor}{rgb}{0.545, 0.0, 0.0}



    
    % Prevent overflowing lines due to hard-to-break entities
    \sloppy 
    % Setup hyperref package
    \hypersetup{
      breaklinks=true,  % so long urls are correctly broken across lines
      colorlinks=true,
      urlcolor=blue,
      linkcolor=darkorange,
      citecolor=darkgreen,
      }
    % Slightly bigger margins than the latex defaults
    
    \geometry{verbose,tmargin=1in,bmargin=1in,lmargin=1in,rmargin=1in}
    
    

    \begin{document}
    
    
    \maketitle
    
    

    
    \section{Regular Expression}\label{regular-expression}

    \textbf{Regular expressions} (called REs, or regexes, or regex patterns)
are essentially a tiny, highly specialized programming language embedded
inside Python and made available through the \texttt{re} module.

The regular expression language is relatively small and restricted, so
not all possible string processing tasks can be done using regular
expressions. There are also tasks that can be done with regular
expressions, but the expressions turn out to be very complicated. In
these cases, you may be better off writing Python code to do the
processing; while Python code will be slower than an elaborate regular
expression, it will also probably be more understandable.

    \subsection{Regular Expression Syntax}\label{regular-expression-syntax}

    Regular expressions can contain both special and ordinary characters.
Most ordinary characters, like `A', `a', or `0', are the simplest
regular expressions; they simply match themselves. You can concatenate
ordinary characters, so last matches the string `last'.

Some characters, like `\textbar{}' or `(', are special. \textbf{Special
characters} either stand for classes of ordinary characters, or affect
how the regular expressions around them are interpreted.

\begin{itemize}
\item
  \texttt{{[}{]}}: Used to indicate a set of characters. In a set:

  \begin{itemize}
  \itemsep1pt\parskip0pt\parsep0pt
  \item
    Characters can be listed individually, e.g. {[}amk{]} will match
    `a', `m', or `k'.
  \item
    Ranges of characters can be indicated by giving two characters and
    separating them by a `-', for example {[}a-z{]} will match any
    lowercase ASCII letter, {[}0-5{]}{[}0-9{]} will match all the
    two-digits numbers from 00 to 59, and {[}0-9A-Fa-f{]} will match any
    hexadecimal digit. If - is escaped (e.g. {[}a-z{]}) or if it's
    placed as the first or last character (e.g. {[}a-{]}), it will match
    a literal `-'.
  \end{itemize}
\item
  \texttt{(...)}: Matches whatever regular expression is inside the
  parentheses, and indicates the start and end of a group; the contents
  of a group can be retrieved after a match has been performed, and can
  be matched later in the string with the
  \texttt{\textbackslash{}number} special sequence, described below. To
  match the literals `(' or `)', use ( or ), or enclose them inside a
  character class: {[}({]} {[}){]}.
\item
  `\texttt{\textbar{}'}: A\textbar{}B, where A and B can be arbitrary
  REs, creates a regular expression that will match either A or B. An
  arbitrary number of REs can be separated by the'\textbar{}`in this
  way. This can be used inside groups (see below) as well. As the target
  string is scanned, REs separated by'\textbar{}`are tried from left to
  right. When one pattern completely matches, that branch is accepted.
  This means that once A matches, B will not be tested further, even if
  it would produce a longer overall match. In other words,
  the'\textbar{}`operator is never greedy. To match a
  literal'\textbar{}', use \textbar{}, or enclose it inside a character
  class, as in {[}\textbar{}{]}.
\end{itemize}

The special sequences consist of `' and a character from the list below.
If the ordinary character is not on the list, then the resulting RE will
match the second character. For example, \$ matches the character'\$'

\begin{itemize}
\item
  \texttt{\textbackslash{}d}: Matches any decimal digit; this is
  equivalent to the class \texttt{{[}0-9{]}}.
\item
  \texttt{\textbackslash{}D}: Matches any non-digit character; this is
  equivalent to the class \texttt{{[}\^{}0-9{]}}.
\item
  \texttt{\textbackslash{}s}: Matches any whitespace character; this is
  equivalent to the class
  \texttt{{[} \textbackslash{}t\textbackslash{}n\textbackslash{}r\textbackslash{}f\textbackslash{}v{]}}.
\item
  \texttt{\textbackslash{}S}: Matches any non-whitespace character; this
  is equivalent to the class
  \texttt{{[}\^{} \textbackslash{}t\textbackslash{}n\textbackslash{}r\textbackslash{}f\textbackslash{}v{]}}.
\item
  \texttt{\textbackslash{}w}: Matches any alphanumeric character; this
  is equivalent to the class \texttt{{[}a-zA-Z0-9\_{]}}.
\item
  \texttt{\textbackslash{}W}: Matches any non-alphanumeric character;
  this is equivalent to the class \texttt{{[}\^{}a-zA-Z0-9\_{]}}.
\end{itemize}

These sequences can be included inside a character class. For example,
\texttt{{[}\textbackslash{}s,.{]}} is a character class that will match
any whitespace character, or `,' or `.'.

The final metacharacter in this section is .. It matches anything except
a newline character, and there's an alternate mode (re.DOTALL) where it
will match even a newline. `.' is often used where you want to match
``any character''.

For a complete list of sequences and expanded class definitions for
Unicode string patterns, see the last part of
\href{https://docs.python.org/3.4/library/re.html\#re-syntax}{Regular
Expression Syntax in the Standard Library} reference.

    \subsection{Python \texttt{re} module}\label{python-re-module}

    This module provides regular expression matching operations similar to
those found in Perl.

    \subsubsection{Compiling Regular
Expressions}\label{compiling-regular-expressions}

    The module defines several functions, constants, and an exception. Some
of the functions are simplified versions of the full featured methods
for compiled regular expressions. Most non-trivial applications always
use the compiled form. Compile a regular expression pattern into a
regular expression object, which can be used for matching using its
\texttt{match()} and \texttt{search()} methods

\begin{verbatim}
    re.compile(pattern, flags=0)
\end{verbatim}

Regular expressions are compiled into pattern objects, which have
methods for various operations such as searching for pattern matches or
performing string substitutions.

    \begin{Verbatim}[commandchars=\\\{\}]
{\color{incolor}In [{\color{incolor}42}]:} \PY{n}{p} \PY{o}{=} \PY{n}{re}\PY{o}{.}\PY{n}{compile}\PY{p}{(}\PY{l+s}{\PYZsq{}}\PY{l+s}{o}\PY{l+s}{\PYZsq{}}\PY{p}{)}
         \PY{n}{p}
\end{Verbatim}

            \begin{Verbatim}[commandchars=\\\{\}]
{\color{outcolor}Out[{\color{outcolor}42}]:} re.compile(r'o', re.UNICODE)
\end{Verbatim}
        
    \begin{Verbatim}[commandchars=\\\{\}]
{\color{incolor}In [{\color{incolor}43}]:} \PY{n+nb}{print}\PY{p}{(}\PY{n}{p}\PY{o}{.}\PY{n}{search}\PY{p}{(}\PY{l+s}{\PYZsq{}}\PY{l+s}{I love Python}\PY{l+s}{\PYZsq{}}\PY{p}{)}\PY{p}{)}
\end{Verbatim}

    \begin{Verbatim}[commandchars=\\\{\}]
<\_sre.SRE\_Match object; span=(3, 4), match='o'>
    \end{Verbatim}

    \subsubsection{Backslash character ('')}\label{backslash-character}

    Regular expressions use the backslash character (`') to indicate special
forms or to allow special characters to be used without invoking their
special meaning. This collides with Python's usage of the same character
for the same purpose in string literals; for example, to match a literal
backslash, one might have to write'\textbackslash{}\textbackslash{}' as
the pattern string, because the regular expression must be
\textbackslash{}, and each backslash must be expressed as
\textbackslash{} inside a regular Python string literal.

    \begin{Verbatim}[commandchars=\\\{\}]
{\color{incolor}In [{\color{incolor}1}]:} \PY{n+nb}{print}\PY{p}{(}\PY{l+s}{\PYZsq{}}\PY{l+s+se}{\PYZbs{}\PYZbs{}}\PY{l+s+se}{\PYZbs{}\PYZbs{}}\PY{l+s}{\PYZsq{}}\PY{p}{)}
\end{Verbatim}

    \begin{Verbatim}[commandchars=\\\{\}]
\textbackslash{}\textbackslash{}
    \end{Verbatim}

    The solution is to use Python's raw string notation for regular
expression patterns; backslashes are not handled in any special way in a
string literal prefixed with `r'. So \texttt{r"\textbackslash{}n"} is a
two-character string containing `' and 'n', while
\texttt{\textbackslash{}n} is a one-character string containing a
newline. Usually patterns will be expressed in Python code using this
raw string notation.

    \begin{Verbatim}[commandchars=\\\{\}]
{\color{incolor}In [{\color{incolor}2}]:} \PY{n+nb}{print}\PY{p}{(}\PY{l+s}{\PYZsq{}}\PY{l+s+se}{\PYZbs{}n}\PY{l+s}{\PYZsq{}}\PY{p}{)} \PY{c}{\PYZsh{} print a new line}
\end{Verbatim}

    \begin{Verbatim}[commandchars=\\\{\}]

    \end{Verbatim}

    \begin{Verbatim}[commandchars=\\\{\}]
{\color{incolor}In [{\color{incolor}3}]:} \PY{n+nb}{print}\PY{p}{(}\PY{l+s}{r\PYZsq{}}\PY{l+s}{\PYZbs{}}\PY{l+s}{n}\PY{l+s}{\PYZsq{}}\PY{p}{)} \PY{c}{\PYZsh{} print \PYZsq{}\PYZbs{}n\PYZsq{} string}
\end{Verbatim}

    \begin{Verbatim}[commandchars=\\\{\}]
\textbackslash{}n
    \end{Verbatim}

    \subsection{Matching Characters}\label{matching-characters}

    \subsection{\texttt{re.match()} and
\texttt{re.search()}}\label{re.match-and-re.search}

    Python offers two different primitive operations based on regular
expressions: \texttt{re.match()} checks for a match only at the
beginning of the string, while \texttt{re.search()} checks for a match
anywhere in the string (this is what Perl does by default).

    \begin{Verbatim}[commandchars=\\\{\}]
{\color{incolor}In [{\color{incolor}9}]:} \PY{k+kn}{import} \PY{n+nn}{re}
        \PY{n}{out1} \PY{o}{=} \PY{n}{re}\PY{o}{.}\PY{n}{match}\PY{p}{(}\PY{l+s}{\PYZsq{}}\PY{l+s}{c}\PY{l+s}{\PYZsq{}}\PY{p}{,} \PY{l+s}{\PYZdq{}}\PY{l+s}{I love coding}\PY{l+s}{\PYZdq{}}\PY{p}{)}
        \PY{n}{out2} \PY{o}{=} \PY{n}{re}\PY{o}{.}\PY{n}{search}\PY{p}{(}\PY{l+s}{\PYZsq{}}\PY{l+s}{c}\PY{l+s}{\PYZsq{}}\PY{p}{,} \PY{l+s}{\PYZdq{}}\PY{l+s}{I love coding}\PY{l+s}{\PYZdq{}}\PY{p}{)}
        
        \PY{n+nb}{print}\PY{p}{(}\PY{n}{out1}\PY{p}{)}
        \PY{n+nb}{print}\PY{p}{(}\PY{n}{out2}\PY{p}{)}
\end{Verbatim}

    \begin{Verbatim}[commandchars=\\\{\}]
None
<\_sre.SRE\_Match object; span=(7, 8), match='c'>
    \end{Verbatim}

    Regular expressions beginning with `\^{}' can be used with
\texttt{search()} to restrict the match at the beginning of the string:

    \begin{Verbatim}[commandchars=\\\{\}]
{\color{incolor}In [{\color{incolor}12}]:} \PY{n+nb}{print}\PY{p}{(}\PY{n}{re}\PY{o}{.}\PY{n}{match}\PY{p}{(}\PY{l+s}{\PYZdq{}}\PY{l+s}{c}\PY{l+s}{\PYZdq{}}\PY{p}{,} \PY{l+s}{\PYZdq{}}\PY{l+s}{abcdef}\PY{l+s}{\PYZdq{}}\PY{p}{)}\PY{p}{)}  \PY{c}{\PYZsh{} No match}
         \PY{n+nb}{print}\PY{p}{(}\PY{n}{re}\PY{o}{.}\PY{n}{search}\PY{p}{(}\PY{l+s}{\PYZdq{}}\PY{l+s}{\PYZca{}c}\PY{l+s}{\PYZdq{}}\PY{p}{,} \PY{l+s}{\PYZdq{}}\PY{l+s}{abcdef}\PY{l+s}{\PYZdq{}}\PY{p}{)}\PY{p}{)} \PY{c}{\PYZsh{} No match}
         \PY{n+nb}{print}\PY{p}{(}\PY{n}{re}\PY{o}{.}\PY{n}{search}\PY{p}{(}\PY{l+s}{\PYZdq{}}\PY{l+s}{\PYZca{}a}\PY{l+s}{\PYZdq{}}\PY{p}{,} \PY{l+s}{\PYZdq{}}\PY{l+s}{abcdef}\PY{l+s}{\PYZdq{}}\PY{p}{)}\PY{p}{)}  \PY{c}{\PYZsh{} Match}
\end{Verbatim}

    \begin{Verbatim}[commandchars=\\\{\}]
None
None
<\_sre.SRE\_Match object; span=(0, 1), match='a'>
    \end{Verbatim}

    In \texttt{MULTILINE} mode \texttt{match()} only matches at the
beginning of the string, whereas using \texttt{search()} with a regular
expression beginning with `\^{}' will match at the beginning of each
line.

    \begin{Verbatim}[commandchars=\\\{\}]
{\color{incolor}In [{\color{incolor}17}]:} \PY{n+nb}{print}\PY{p}{(}\PY{n}{re}\PY{o}{.}\PY{n}{match}\PY{p}{(}\PY{l+s}{\PYZsq{}}\PY{l+s}{X}\PY{l+s}{\PYZsq{}}\PY{p}{,} \PY{l+s}{\PYZsq{}}\PY{l+s}{A}\PY{l+s+se}{\PYZbs{}n}\PY{l+s}{B}\PY{l+s+se}{\PYZbs{}n}\PY{l+s}{X}\PY{l+s}{\PYZsq{}}\PY{p}{,} \PY{n}{re}\PY{o}{.}\PY{n}{MULTILINE}\PY{p}{)}\PY{p}{)}  \PY{c}{\PYZsh{} No match}
         \PY{n+nb}{print}\PY{p}{(}\PY{n}{re}\PY{o}{.}\PY{n}{search}\PY{p}{(}\PY{l+s}{\PYZsq{}}\PY{l+s}{\PYZca{}X}\PY{l+s}{\PYZsq{}}\PY{p}{,} \PY{l+s}{\PYZsq{}}\PY{l+s}{A}\PY{l+s+se}{\PYZbs{}n}\PY{l+s}{B}\PY{l+s+se}{\PYZbs{}n}\PY{l+s}{X}\PY{l+s}{\PYZsq{}}\PY{p}{,} \PY{n}{re}\PY{o}{.}\PY{n}{MULTILINE}\PY{p}{)}\PY{p}{)}  \PY{c}{\PYZsh{} Match}
\end{Verbatim}

    \begin{Verbatim}[commandchars=\\\{\}]
None
<\_sre.SRE\_Match object; span=(4, 5), match='X'>
    \end{Verbatim}

    \subsection{\texttt{match.group({[}group1, ...{]})}}\label{match.groupgroup1-...}

Returns one or more subgroups of the match. If there is a single
argument, the result is a single string; if there are multiple
arguments, the result is a tuple with one item per argument. Without
arguments, group1 defaults to zero (the whole match is returned). If a
groupN argument is zero, the corresponding return value is the entire
matching string; if it is in the inclusive range \texttt{{[}1..99{]}},
it is the string matching the corresponding parenthesized group. If a
group number is negative or larger than the number of groups defined in
the pattern, an IndexError exception is raised. If a group is contained
in a part of the pattern that did not match, the corresponding result is
None. If a group is contained in a part of the pattern that matched
multiple times, the last match is returned.

    \begin{Verbatim}[commandchars=\\\{\}]
{\color{incolor}In [{\color{incolor}68}]:} \PY{n}{m} \PY{o}{=} \PY{n}{re}\PY{o}{.}\PY{n}{match}\PY{p}{(}\PY{l+s}{r\PYZdq{}}\PY{l+s}{(}\PY{l+s}{\PYZbs{}}\PY{l+s}{w+) (}\PY{l+s}{\PYZbs{}}\PY{l+s}{w+)}\PY{l+s}{\PYZdq{}}\PY{p}{,} \PY{l+s}{\PYZdq{}}\PY{l+s}{Isaac Newton, physicist}\PY{l+s}{\PYZdq{}}\PY{p}{)}
         \PY{n+nb}{print}\PY{p}{(}\PY{n}{m}\PY{o}{.}\PY{n}{group}\PY{p}{(}\PY{l+m+mi}{0}\PY{p}{)}\PY{p}{)}       \PY{c}{\PYZsh{} The entire match}
         
         \PY{n+nb}{print}\PY{p}{(}\PY{n}{m}\PY{o}{.}\PY{n}{group}\PY{p}{(}\PY{l+m+mi}{1}\PY{p}{)}\PY{p}{)}       \PY{c}{\PYZsh{} The first parenthesized subgroup.}
         
         \PY{n+nb}{print}\PY{p}{(}\PY{n}{m}\PY{o}{.}\PY{n}{group}\PY{p}{(}\PY{l+m+mi}{2}\PY{p}{)}\PY{p}{)}       \PY{c}{\PYZsh{} The second parenthesized subgroup.}
         
         \PY{n+nb}{print}\PY{p}{(}\PY{n}{m}\PY{o}{.}\PY{n}{group}\PY{p}{(}\PY{l+m+mi}{1}\PY{p}{,} \PY{l+m+mi}{2}\PY{p}{)}\PY{p}{)}    \PY{c}{\PYZsh{} Multiple arguments give us a tuple.}
\end{Verbatim}

    \begin{Verbatim}[commandchars=\\\{\}]
Isaac Newton
Isaac
Newton
('Isaac', 'Newton')
    \end{Verbatim}

    \subsection{\texttt{match.start({[}group{]})} and
\texttt{match.end({[}group{]})}}\label{match.startgroup-and-match.endgroup}

Return the indices of the start and end of the substring matched by
group; group defaults to zero (meaning the whole matched substring).
Return -1 if group exists but did not contribute to the match. For a
match object m, and a group g that did contribute to the match, the
substring matched by group g (equivalent to m.group(g)) is

    \begin{Verbatim}[commandchars=\\\{\}]
{\color{incolor}In [{\color{incolor}46}]:} \PY{n}{email} \PY{o}{=} \PY{l+s}{\PYZdq{}}\PY{l+s}{tony@tiremove\PYZus{}thisger.net}\PY{l+s}{\PYZdq{}}
         \PY{n}{m} \PY{o}{=} \PY{n}{re}\PY{o}{.}\PY{n}{search}\PY{p}{(}\PY{l+s}{\PYZdq{}}\PY{l+s}{remove\PYZus{}this}\PY{l+s}{\PYZdq{}}\PY{p}{,} \PY{n}{email}\PY{p}{)}
         \PY{n}{email}\PY{p}{[}\PY{p}{:}\PY{n}{m}\PY{o}{.}\PY{n}{start}\PY{p}{(}\PY{p}{)}\PY{p}{]} \PY{o}{+} \PY{n}{email}\PY{p}{[}\PY{n}{m}\PY{o}{.}\PY{n}{end}\PY{p}{(}\PY{p}{)}\PY{p}{:}\PY{p}{]}
\end{Verbatim}

            \begin{Verbatim}[commandchars=\\\{\}]
{\color{outcolor}Out[{\color{outcolor}46}]:} 'tony@tiger.net'
\end{Verbatim}
        
    \subsection{Splitting Strings}\label{splitting-strings}

    The \texttt{split()} method of a pattern splits a string apart wherever
the RE matches, returning a list of the pieces. It's similar to the
\texttt{split()} method of strings but provides much more generality in
the delimiters that you can split by; string \texttt{split()} only
supports splitting by whitespace or by a fixed string.

    \begin{Verbatim}[commandchars=\\\{\}]
{\color{incolor}In [{\color{incolor}44}]:} \PY{n}{re}\PY{o}{.}\PY{n}{split}\PY{p}{(}\PY{l+s}{\PYZsq{}}\PY{l+s}{\PYZbs{}}\PY{l+s}{W+}\PY{l+s}{\PYZsq{}}\PY{p}{,} \PY{l+s}{\PYZsq{}}\PY{l+s}{Words, words, words.}\PY{l+s}{\PYZsq{}}\PY{p}{)}
\end{Verbatim}

            \begin{Verbatim}[commandchars=\\\{\}]
{\color{outcolor}Out[{\color{outcolor}44}]:} ['Words', 'words', 'words', '']
\end{Verbatim}
        
    \begin{Verbatim}[commandchars=\\\{\}]
{\color{incolor}In [{\color{incolor}47}]:} \PY{n}{re}\PY{o}{.}\PY{n}{split}\PY{p}{(}\PY{l+s}{\PYZsq{}}\PY{l+s}{(}\PY{l+s}{\PYZbs{}}\PY{l+s}{W+)}\PY{l+s}{\PYZsq{}}\PY{p}{,} \PY{l+s}{\PYZsq{}}\PY{l+s}{Words, words, words.}\PY{l+s}{\PYZsq{}}\PY{p}{)}
\end{Verbatim}

            \begin{Verbatim}[commandchars=\\\{\}]
{\color{outcolor}Out[{\color{outcolor}47}]:} ['Words', ', ', 'words', ', ', 'words', '.', '']
\end{Verbatim}
        
    \begin{Verbatim}[commandchars=\\\{\}]
{\color{incolor}In [{\color{incolor}48}]:} \PY{n}{re}\PY{o}{.}\PY{n}{split}\PY{p}{(}\PY{l+s}{\PYZsq{}}\PY{l+s}{\PYZbs{}}\PY{l+s}{W+}\PY{l+s}{\PYZsq{}}\PY{p}{,} \PY{l+s}{\PYZsq{}}\PY{l+s}{Words, words, words.}\PY{l+s}{\PYZsq{}}\PY{p}{,} \PY{l+m+mi}{1}\PY{p}{)}
\end{Verbatim}

            \begin{Verbatim}[commandchars=\\\{\}]
{\color{outcolor}Out[{\color{outcolor}48}]:} ['Words', 'words, words.']
\end{Verbatim}
        
    \begin{Verbatim}[commandchars=\\\{\}]
{\color{incolor}In [{\color{incolor}49}]:} \PY{n}{re}\PY{o}{.}\PY{n}{split}\PY{p}{(}\PY{l+s}{\PYZsq{}}\PY{l+s}{[a\PYZhy{}f]+}\PY{l+s}{\PYZsq{}}\PY{p}{,} \PY{l+s}{\PYZsq{}}\PY{l+s}{0a3B9}\PY{l+s}{\PYZsq{}}\PY{p}{,} \PY{n}{flags}\PY{o}{=}\PY{n}{re}\PY{o}{.}\PY{n}{IGNORECASE}\PY{p}{)}
\end{Verbatim}

            \begin{Verbatim}[commandchars=\\\{\}]
{\color{outcolor}Out[{\color{outcolor}49}]:} ['0', '3', '9']
\end{Verbatim}
        
    \subsection{Substitution}\label{substitution}

\begin{verbatim}
 re.sub(pattern, repl, string, count=0, flags=0)
\end{verbatim}

Return the string obtained by replacing the leftmost non-overlapping
occurrences of pattern in string by the replacement repl. If the pattern
isn't found, string is returned unchanged. repl can be a string or a
function; if it is a string, any backslash escapes in it are processed.
That is, \texttt{\textbackslash{}n} is converted to a single newline
character, \texttt{\textbackslash{}r} is converted to a carriage return,
and so forth. Unknown escapes such as \texttt{\textbackslash{}j} are
left alone. Backreferences, such as \texttt{\textbackslash{}6}, are
replaced with the substring matched by group 6 in the pattern. For
example:

    \begin{Verbatim}[commandchars=\\\{\}]
{\color{incolor}In [{\color{incolor}66}]:} \PY{n}{re}\PY{o}{.}\PY{n}{sub}\PY{p}{(}\PY{l+s}{r\PYZsq{}}\PY{l+s}{def}\PY{l+s}{\PYZbs{}}\PY{l+s}{s+([a\PYZhy{}zA\PYZhy{}Z\PYZus{}][a\PYZhy{}zA\PYZhy{}Z\PYZus{}0\PYZhy{}9]*)}\PY{l+s}{\PYZbs{}}\PY{l+s}{s*}\PY{l+s}{\PYZbs{}}\PY{l+s}{(}\PY{l+s}{\PYZbs{}}\PY{l+s}{s*}\PY{l+s}{\PYZbs{}}\PY{l+s}{):}\PY{l+s}{\PYZsq{}}\PY{p}{,}
                \PY{l+s}{r\PYZsq{}}\PY{l+s}{static PyObject*}\PY{l+s}{\PYZbs{}}\PY{l+s}{npy\PYZus{}}\PY{l+s}{\PYZbs{}}\PY{l+s}{1(void)}\PY{l+s}{\PYZbs{}}\PY{l+s}{n\PYZob{}}\PY{l+s}{\PYZsq{}}\PY{p}{,}
                \PY{l+s}{\PYZsq{}}\PY{l+s}{def myfunc():}\PY{l+s}{\PYZsq{}}\PY{p}{)}
\end{Verbatim}

            \begin{Verbatim}[commandchars=\\\{\}]
{\color{outcolor}Out[{\color{outcolor}66}]:} 'static PyObject*\textbackslash{}npy\_myfunc(void)\textbackslash{}n\{'
\end{Verbatim}
        
    \section{Classes}\label{classes}

Compared with other programming languages, Python's class mechanism adds
classes with a minimum of new syntax and semantics. Python classes
provide all the standard features of Object Oriented Programming:

\begin{itemize}
\itemsep1pt\parskip0pt\parsep0pt
\item
  the class inheritance mechanism allows multiple base classes,
\item
  a derived class can override any methods of its base class or classes,
  and
\item
  a method can call the method of a base class with the same name.
\end{itemize}

Class definitions, like function definitions (\texttt{def} statements)
must be executed before they have any effect. (You could conceivably
place a class definition in a branch of an if statement, or inside a
function.)

In practice, the statements inside a class definition will usually be
function definitions, but other statements are allowed, and sometimes
useful. The function definitions inside a class normally have a peculiar
form of argument list, dictated by the calling conventions for methods.

When a class definition is entered, a new \textbf{namespace} is created,
and used as the \emph{local scope} --- thus, all assignments to local
variables go into this new namespace. In particular, function
definitions bind the name of the new function here.

Often, the first argument of a method is called self. This is nothing
more than a convention: the name self has absolutely no special meaning
to Python. Note, however, that by not following the convention your code
may be less readable to other Python programmers, and it is also
conceivable that a class browser program might be written that relies
upon such a convention.

When a class definition is left normally (via the end), a class object
is created. This is basically a wrapper around the contents of the
namespace created by the class definition; The original local scope (the
one in effect just before the class definition was entered) is
reinstated, and the class object is bound here to the class name given
in the class definition header. Objects can contain arbitrary amounts
and kinds of data. As is true for modules, classes partake of the
dynamic nature of Python: \emph{they are created at runtime, and can be
modified further after creation}.

\begin{itemize}
\itemsep1pt\parskip0pt\parsep0pt
\item
  \href{https://docs.python.org/3/tutorial/classes.html}{Python
  tutorial: Classes}
\end{itemize}

    \subsection{Class Objects}\label{class-objects}

Class objects support two kinds of operations: \textbf{attribute
references} and \textbf{instantiation}.

    \begin{Verbatim}[commandchars=\\\{\}]
{\color{incolor}In [{\color{incolor}50}]:} \PY{k}{class} \PY{n+nc}{MyClass}\PY{p}{:}
             \PY{l+s+sd}{\PYZdq{}\PYZdq{}\PYZdq{}A simple example class\PYZdq{}\PYZdq{}\PYZdq{}}
             \PY{n}{i} \PY{o}{=} \PY{l+m+mi}{12345}
             \PY{k}{def} \PY{n+nf}{f}\PY{p}{(}\PY{n+nb+bp}{self}\PY{p}{)}\PY{p}{:}
                 \PY{k}{return} \PY{l+s}{\PYZsq{}}\PY{l+s}{hello world}\PY{l+s}{\PYZsq{}}
\end{Verbatim}

    \begin{Verbatim}[commandchars=\\\{\}]
{\color{incolor}In [{\color{incolor}51}]:} \PY{n}{MyClass}\PY{p}{(}\PY{p}{)}
\end{Verbatim}

            \begin{Verbatim}[commandchars=\\\{\}]
{\color{outcolor}Out[{\color{outcolor}51}]:} <\_\_main\_\_.MyClass at 0x7f195922e898>
\end{Verbatim}
        
    \textbf{Attribute references} use the standard syntax used for all
attribute references in Python: \texttt{obj.name}. Valid attribute names
are all the names that were in the class's namespace when the class
object was created.

    \begin{Verbatim}[commandchars=\\\{\}]
{\color{incolor}In [{\color{incolor}52}]:} \PY{n}{MyClass}\PY{o}{.}\PY{n}{i}
\end{Verbatim}

            \begin{Verbatim}[commandchars=\\\{\}]
{\color{outcolor}Out[{\color{outcolor}52}]:} 12345
\end{Verbatim}
        
    \begin{Verbatim}[commandchars=\\\{\}]
{\color{incolor}In [{\color{incolor}53}]:} \PY{n}{MyClass}\PY{o}{.}\PY{n}{f}
\end{Verbatim}

            \begin{Verbatim}[commandchars=\\\{\}]
{\color{outcolor}Out[{\color{outcolor}53}]:} <function \_\_main\_\_.MyClass.f>
\end{Verbatim}
        
    \texttt{MyClass.i} and \texttt{MyClass.f} are valid attribute
references, returning an integer and a function object, respectively.
Class attributes can also be assigned to, so you can change the value of
\texttt{MyClass.i} by assignment.

    \begin{Verbatim}[commandchars=\\\{\}]
{\color{incolor}In [{\color{incolor}55}]:} \PY{n}{MyClass}\PY{o}{.}\PY{n}{i} \PY{o}{=} \PY{l+m+mi}{3}
\end{Verbatim}

    \begin{Verbatim}[commandchars=\\\{\}]
{\color{incolor}In [{\color{incolor}56}]:} \PY{n}{MyClass}\PY{o}{.}\PY{n}{i}
\end{Verbatim}

            \begin{Verbatim}[commandchars=\\\{\}]
{\color{outcolor}Out[{\color{outcolor}56}]:} 3
\end{Verbatim}
        
    The \textbf{instantiation operation} (``calling'' a class object)
creates an empty object. Many classes like to create objects with
instances customized to a specific initial state. Therefore a class may
define a special method named \texttt{\_\_init\_\_()}, like this:

    \begin{Verbatim}[commandchars=\\\{\}]
{\color{incolor}In [{\color{incolor}64}]:} \PY{k}{def} \PY{n+nf}{\PYZus{}\PYZus{}init\PYZus{}\PYZus{}}\PY{p}{(}\PY{n+nb+bp}{self}\PY{p}{)}\PY{p}{:}
             \PY{n+nb+bp}{self}\PY{o}{.}\PY{n}{data} \PY{o}{=} \PY{p}{[}\PY{p}{]}
\end{Verbatim}

    When a class defines an \texttt{\_\_init\_\_()} method, class
instantiation automatically invokes \texttt{\_\_init\_\_()} for the
newly-created class instance. So in this example, a new, initialized
instance can be obtained by calling

    \begin{Verbatim}[commandchars=\\\{\}]
{\color{incolor}In [{\color{incolor}63}]:} \PY{n}{x} \PY{o}{=} \PY{n}{MyClass}\PY{p}{(}\PY{p}{)}
         \PY{n+nb}{print}\PY{p}{(}\PY{n}{x}\PY{p}{)}
\end{Verbatim}

    \begin{Verbatim}[commandchars=\\\{\}]
<\_\_main\_\_.MyClass object at 0x7f19591b4da0>
    \end{Verbatim}

    Of course, the \texttt{\_\_init\_\_()} method may have arguments for
greater flexibility. In that case, arguments given to the class
instantiation operator are passed on to \texttt{\_\_init\_\_()}.

    \begin{Verbatim}[commandchars=\\\{\}]
{\color{incolor}In [{\color{incolor}59}]:} \PY{k}{class} \PY{n+nc}{Complex}\PY{p}{:}
             \PY{k}{def} \PY{n+nf}{\PYZus{}\PYZus{}init\PYZus{}\PYZus{}}\PY{p}{(}\PY{n+nb+bp}{self}\PY{p}{,} \PY{n}{realpart}\PY{p}{,} \PY{n}{imagpart}\PY{p}{)}\PY{p}{:}
                 \PY{n+nb+bp}{self}\PY{o}{.}\PY{n}{r} \PY{o}{=} \PY{n}{realpart}
                 \PY{n+nb+bp}{self}\PY{o}{.}\PY{n}{i} \PY{o}{=} \PY{n}{imagpart}
         
         \PY{n}{x} \PY{o}{=} \PY{n}{Complex}\PY{p}{(}\PY{l+m+mf}{3.0}\PY{p}{,} \PY{o}{\PYZhy{}}\PY{l+m+mf}{4.5}\PY{p}{)}
         \PY{n}{x}\PY{o}{.}\PY{n}{r}\PY{p}{,} \PY{n}{x}\PY{o}{.}\PY{n}{i}
\end{Verbatim}

            \begin{Verbatim}[commandchars=\\\{\}]
{\color{outcolor}Out[{\color{outcolor}59}]:} (3.0, -4.5)
\end{Verbatim}
        
    \section{Web Scraping}\label{web-scraping}

One of the challenges of writing web crawlers is that you're often
performing the same tasks again and again: find all links on a page,
evaluate the difference between internal and external links, go to new
pages. These basic patterns are useful to know about and to be able to
write from scratch, but there are options if you want something else to
handle the details for you.

\textbf{Scrapy} is a Python library that handles much of the complexity
of finding and evaluating links on a website, crawling domains or lists
of domains with ease. Unfortunately, Scrapy has not yet been released
for Python 3.x, though it is compatible with Python 2.7.

    \subsection{Create a Scrapy project}\label{create-a-scrapy-project}

Although writing Scrapy crawlers is relatively easy, there is a small
amount of setup that needs to be done for each crawler. To create a new
Scrapy project in the current directory, run from the command line:

\begin{verbatim}
scrapy startproject wikiSpider
\end{verbatim}

    In order to create a crawler, we will add a new file to
\texttt{wikiSpider/wikiSpider/spiders/} called \texttt{items.py} . In
addition, we will define a new item called \texttt{Article} inside the
\texttt{items.py} file.

    Your \texttt{items.py} file should be edited to look like this (with
Scrapy-generated comments left in place, although you can feel free to
remove them):

\begin{verbatim}
# -*- coding: utf-8 -*-
# Define here the models for your scraped items
# See documentation in:
# http://doc.scrapy.org/en/latest/topics/items.html
from scrapy import Item, Field

class Article(Item):
        # define the fields for your item here like:
        # name = scrapy.)
        title = Field()
\end{verbatim}

    Now you create a file
\texttt{wikiSpider/wikiSpider/spiders/articleSpider.py}. In your newly
created \texttt{articleSpider.py} file, write the following

\begin{verbatim}
# -*- coding: utf-8 -*-
from scrapy.selector import Selector
from scrapy import Spider
from wikiSpider.items import Article
class ArticleSpider(Spider):
    name="article"
    allowed_domains = ["en.wikipedia.org"]
    start_urls = ["http://en.wikipedia.org/wiki/Main_Page",
                  "http://en.wikipedia.org/wiki/Python_%28programming_language%29"]
    def parse(self, response):
        item = Article()
        title = response.xpath('//h1/text()')[0].extract()

        item['title'] = title
        yield item
\end{verbatim}

    Then go to your \texttt{wikiSpider} project home directory and run the
project

\begin{verbatim}
scrapy crawl article -o wiki.csv
\end{verbatim}

The information is now saved in \texttt{wiki.csv} file.

    \subsection{Retrieve a table from web}\label{retrieve-a-table-from-web}

Still the \texttt{Python Programming Language} topic on Wikipedia, we'd
like to extract the talbe entitled ``Summary of Python 3's built-in
types''. First, let's create a new project called
\texttt{wikiPythonTable}

\begin{verbatim}
scrapy startproject wikiPythonTable
\end{verbatim}

    And we add more items to the \texttt{items.py} file under
\texttt{wikiPythonTable/wikiPythonTable/}

\begin{verbatim}
# -*- coding: utf-8 -*-
# Define here the models for your scraped items
# See documentation in:
# http://doc.scrapy.org/en/latest/topics/items.html
from scrapy import Item, Field

class Article(Item):
        # define the fields for your item here like:
        # name = scrapy.)
        datatype = Field()
        mutable = Field()
        description = Field()
        syntax = Field()
\end{verbatim}

    We now need a \texttt{tableSpider.py} under
\texttt{wikiPythonTable/wikiPythonTable/spider}

\begin{verbatim}
# -*- coding: utf-8 -*-
from scrapy.selector import Selector
from scrapy import Spider
from wikiPythonTable.items import Article
class WikiPythonTable(Spider):
    name="table"
    allowed_domains = ["en.wikipedia.org"]
    start_urls = ["http://en.wikipedia.org/wiki/Python_%28programming_language%29"]
    def parse(self, response):
        item = Article()

        table_path = response.xpath('//table[re:test(@class,"wikitable")]/tr')
        for i in range(1, len(table_path)):
            item['datatype'] = table_path[i].xpath('.//td[1]/code/text()').extract()[0]
            item['mutable'] = table_path[i].xpath('.//td[2]/descendant::text()').extract()
            item['description'] = ''.join(table_path[i].xpath('.//td[3]/descendant::text()').extract())
            item['syntax'] = ' '.join(table_path[i].xpath('.//td[4]/descendant::text()').extract())    
            yield item
\end{verbatim}

    Run your project with at the project root directory

\begin{verbatim}
scrapy crawl table -o table.csv  --logfile table.log
\end{verbatim}

    \subsection{Scrapy Shell}\label{scrapy-shell}

To do the crawler interactively, just run

\begin{verbatim}
scrapy shell "http://en.wikipedia.org/wiki/Python_%28programming_language%29"
\end{verbatim}


    % Add a bibliography block to the postdoc
    
    
    
    \end{document}
